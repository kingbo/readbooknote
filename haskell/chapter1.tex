\chapter{入门指南}
作为本书的第一章,紧记我们有时会用严格的、简洁的形式引入概念。Haskell是一本深奥的语言,突然呈现的方方面面的主题绝大多数都能够证明这一点。为了建立牢固的Haskell基础,我们将对每一个主题进行详细的阐述。
\section{Haskell环境}
Haskell有很多实现方式,其中两种方式应用最为广泛。Hugs解释器主要用于教学。在实际应用中,Glasgow Haskell Complier(GHC)则更为流行。与Hugs相比,GHC更适合于“实际工作":它可以编译本机代码,支持并行执行,提供有用的性能分析和调试工具。基于上述理由,GHC这个Haskell实现将贯穿于本书始终。

GHC有三个主要部件:

\begin{description}
\item[ghc]
快速本机代码的优化编译器。
\item[ghci]
交互式解释器和调试器
\item[runghc]
以脚本的方式运行Haskell程序,而不需要先编译程序。
\end{description}

我们假设你所使用的GHC版本至少是6.8.2,它是2007年推出的稳定版本。本书的很多例子能够不经修改的运行于旧版本之上。然而,我们推荐你的平台上有可用的最新版本。如果你使用Windows或者Mac OS X,你可以轻松快速地使用安装程序。要获得这类平台的GHC拷贝,请访问GHC下载页面(http://www.haskell.org/ghc/download.html),并查找相应的二进制代码包和安装程序。

更多地关于GHC在不同平台上安装的信息,请参阅附录A。

\section{解释器ghci指南}
GHC的交互解释器是一个被称为ghci的程序。这个程序让我们可以输入并检验Haskell 表达式,检验究模块,调试代码。

\endinput