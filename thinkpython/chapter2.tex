\chapter{变量,表达式和语句}
\section{值与类型}
值是编程过程最基本的事物,就象字母或者数字一样。迄今为止我们所看到的值是‘1’、‘2’和‘Hello,World!’

这些值属于不同的类型:2是一个整数,‘Hello, World!’是字符串,之所以这样称呼是因为它由字符所构成的一个字串。你能够识别字符串是因为它被引号隔开了。

输出语句也能工作于整数。
\begin{lstlisting}
>>> print 4
4
\end{lstlisting}
如果你不能够确定一个值的类型是什么,解释器也能够告诉你。

\begin{lstlisting}
>>> type('Hello, World!')
<type 'str'>
>>> type(17)
<type 'int'>
\end{lstlisting}

毋庸置疑,字符串属于str类型,整数属于int类型。带小数点的数属于float型则不那么明显,是因为这类数用浮点形式表示的缘故。

\begin{lstlisting}
>>>type(3.2)
<type 'float'>
\end{lstlisting}

而像‘17’和‘3.2’是什么值类型呢?它们看起来像数字,而它们用单引号括起来的字符串。
\begin{lstlisting}
>>>type('17')
<type 'str'>
>>>type('3.2')
<type 'str'>
\end{lstlisting}
实际上是字符串类型。

当输入诸如1,000,000这样很大的整数时,你可能会尝试用逗号将其分为三位数为一组。在Python中,这不合法的整数,而这样做却是合法的:
\begin{lstlisting}
>>>print 1,000,000
1 0 0
\end{lstlisting}
哦,那根本不是我们所期望的。Python将1,000,000解释为一个用逗号分隔的整数序列,并用空白分隔的方式打印之。
这是我们所见到的第一个语义错误:代码运行过程中不产生任何错误信息,但它不却不做期望的事。
\section{变量}
编程语言最强大的特征之一就是具有操作变量的能力。变量是与之相关联的值的命名。

赋值语句用于创建一个新的变量并给他们赋予一个值:
\begin{lstlisting}
>>>message='And now for something completely different'
>>>n=17
>>>pi=3.1415926535897931
\end{lstlisting}
这个例子创建了三个指派。第一个为新变量名message赋了一个字符串;第二个为n赋了17的值;第三个把$\pi$的(近似)值赋给了变量pi。

在纸上表示变量的通用方式是写出变量名并用箭头指向变量的值。这样的图称称作状态图,主要由于它显示了每一个变量当前的状态。下面的图显示了前面例子的结果:

要显示变量的值,可能使用print语句:
\begin{lstlisting}
>>print n
17
print pi
3.14159265359
\end{lstlisting}
变量的类型是与它相关的值的类型。
\begin{lstlisting}
>>>type(message)
<type 'str'>
>>>type(n)
<type 'int'>
>>>type(pi)
<type 'float'>
\end{lstlisting}

\begin{exercise}
	如果你输入一个以零开头的整数,将获得令人迷惑的错误。
\end{exercise}

\begin{lstlisting}
>>>zipcode=02492
SyntaxError: invalid token
\end{lstlisting}
\endinput