\chapter{编程的方式}
本书的目标是教会你像计算机科学家一样思考。这种思考方式结合了数学、工程以及自然科学最有效的特点。如同数学一样,计算机科学家用形式
化语言表达思想(具体的计算)。像工程一样,他们设计事物,将组件装配成系统以及在多个选择中进行评估权衡。像科学家一样,他们观察复杂
系统的行为,形成假设,并测试预测。

对于计算机科学家而言最重的技能是解决问题。解决问题意味着形式化问题、考虑创造性地大致的解以及更清楚和精确地描述一个解的能力。其结
果是:学习编程的过程是实践解决问题技能的绝佳机会。因此将本章称为“编程方法”。
在一个层面上,你将学习与编程本身相关的实用技能。而另一个层面上,你将用编程作为达到目的的手段。随着我们的前进,结果将变得更加清晰。
\section{Python 编程语言}
%\indent
你将要学习的编程语言是Python。Python是高级语言的一种,你所听到的其它高级是C、C++、Perl以及Java。

还有一些被称作“机器语言”和汇编语言的低级语言。一般来说,计算机仅能执行用低级语言编写的程序。因此用高级语言编写的程序在运行前必须先进行处理。这些额外的处理过程需要花费一定的时间,这也是高级语言的不足之处。

优点是众多的。首先,用高级语言更容易编写程序。用高级语言编写程序比较省时,短小且易读,正确率更高。其次,高级语言是可移植的,意味着它只需经过很少的修改或者不经修改就能够在不同类型的计算机上运行。低级语言编写的程序只能够在同一类型的计算机上运行,而要运行于其它类型的计算机上则不得不进行重新修改。

由于这些优势,大多数的程序都是用高级语言编写的,低级语言仅用于少数的特殊应用。

主要有两类程序将高级语言处理成为低级语言:解释和编译。解释程序读取高级语言程序并执行,意味着程序怎么说它就怎么做。解释程序用很少的时间处理程序,并交替进行程序行读取和执行计算。

编译程序在程序运行前读入程序并将其完全转换。就此而言,高级程序被称为源代码,转换的程序被称为目标代码或者可执行文件。一旦程序被编译,你可以重复的执行它,而无需转换。由于Python程序通过解释器进行执行,所以Python被认为是解释性语言。主要有两种方式使用解释器:交互模式和脚本模式。在交互模式下,你输入Python程序,解释器打印出结果:
\begin{lstlisting}
>>> 1+1
2
\end{lstlisting}
符号\lstinline$>>>$用于提示解释器已经准备好了。如果你输入\lstinline$1+1$,解释器回答2。

此外,你可将代码存放在一个被称为脚本的文件中,并用解释器执行文件中的内容。习惯上,Python的脚本文件以.py作为后缀进行命名。

为了执行这人脚本文件,你应当告诉解释器文件的名称。在UNIX命令窗口中,你应输入python dinsdale.py。在其它开发环境中,执行脚本的细节是不同的。你可以在Python的官方网站python.org上寻找你自己的环境的使用说。

工作在交互下,对于小量的代码是很便利的,因为你能够输入并直接执行他们。要是还需要更的行,就应当将代码保存为脚本,以便于能够进行多次的修改和执行。

\section{什么是编程}
程序是具体说明如何执行计算的指令序列。计算可能是数值的,诸如方程组求解或者多项式求根;也可以是符号计算,例如在一个文档中查找和替换文本或者编译一段程序。

不同语言的细节看起来不同,但是几个指令出现在几乎所有的程序语言之中:
\begin{description}
\item[输入:]从键盘、文件或者其它设备输入数据。
\item[输出:]在屏幕上显示数据或者将数据送往文件或者其它设备。
\item[数值计算:]执行像“加”和“乘”这样的基本数学操作。
\item[条件控制:]检查确定的条件并执行所期望的语句序列。
\item[重复:]重复地执行某些动作,通常伴随着某些变化。
\end{description}

信不信由你,大致就是这么简单。我们所使用的每一个程序,不管它是如何编译的基本上都是由上述指令所构成的。因此,你可以将编程看作这样一个过程:即将巨大的、复杂的任务分解成更小的子任务,直到这些子任务能够用某个基本指令来完成为止。

这也许还有些含糊不清,当我们讨论算法的时候我们还会再回到这里的。
\section{什么是调试}
编程是容易出错的。由于异想天开的理由,程序错误被称为bug并把查找这些错误的过程称为调试。

三类错误经常在程序中出现:语法错误、运行时错误和语义错误。为了更快的查找出错误,仔细区分三类错是很有必要的。

\subsection{语法错误}
Python只能执行语法正确的程序,否则解释器将显示错误信息。语法涉及到程序的结构以及结构的规则。例如:括号必须是成对出现,因此 \lstinline$(1+2)$是合法的,而\lstinline$8)$是有语法错误的。

使用英语的读者能够容忍大多数的语法错误,因而不会在阅读印刷上偏心的诗是冒出错误信息。Python没有如此宽容。如果在你的程序中任何地方有一个错误存在,Python将显示错误信息并退出,并且你将不能够运行你的程序。在你编程生涯的头几个星期时间里,你也许要花很多时间来处理语法错误。当你获得经验后,你将产生更少的语法错误并能够更快的找到这些错误。
\subsection{运行时错误}
第二类错误是运行时错误,之所以这样称呼是因为错误直到程序运行时才出现。由于这些错误通常表示异常发生的事情,因此它们也被称为异常。在开始的几章里你将会明白运行时错误很少出现在简单的程序之中,因此你遇到到这类错误还需要一些时间。
\subsection{语义错误}
第三类错误是语义错误。如果在你的程序中存在语义错误,程序将能够成功地运行,并且计算机不会生成任何错误信息,但是这个程序不能够做正确的事。它将做别的事情,确切的说,你告诉他怎么做他就怎么做。

问题在于你所编写的程序不是你所想编写的程序。这意味着程序(它的语义)是错误的。能够通过跟踪的方式来确定语义错误,因为这要求你通过查看程序的输出来向后查找并试图弄明白程序做了什么。

\subsection{验证性调试}
你将获得的最重要的技能是调试。尽管令人沮丧,但调试是编程中最富于智力、最具挑战和最有趣的部分。

在某些方面,调试很像是侦测工作。面对提示,你必须推断出导致该结果的过程或者事件。

调试也很像实验科学。一旦你有一个什么地方不对劲的想法,你将修改你的程序并再次尝试。如果假设是正确的,你将能够预测修改的结果,并向能够工作的程序靠近了一步。如果假设是错误的,你必须提出一个新的假设。如同谢尔洛克$\cdot$福尔摩斯所说的一样,“当你排除所有的不可能的时候,无论乘下什么,尽管不可,那必定是真的。”
对某些人而言,编程与调试是相同的事。即是说,编程是逐步地调试一个程序直到它做你想要它做的事为此的过程。其思想是:开始于一个能够做一些事的程序,并在此基础上做一些小小的修改,调试并让其通过,以便于有一个始终可以工作的程序。

例如,Linux是一个由成千上万行代码组成的操作系统,但它却开始于一个Linus Torvalds用来搜索Intel 80386芯片的简单程序。据Larry Greenfield说:“Linus最早的方案是用于在AAAA或者BBBB中进行选择的程序,但最终去发展成为Linux。”

在随后的章节将给出更多的关于调试和其它编程实践的建议。

\section{形式语言与自然语言}
\textbf{自然语言}是人们平常交流的语言,诸如:英语、西班牙语、法语。他们不是人们设计的(尽管人们给其增加了不少的规则),而是自然发展而来的。

\textbf{形式语言}是人们为了确切的应用而设计的语言。例如,数学家使用的标记是特别好的用于表示数与符号之间关系的形式语言。化学家使用形式语言来表示化学模型的结构。而最重要的是:

\textbf{编程语言是被设计成为表述计算的形式语言。}

形式语言在语法上倾向于严格的规则。例如:$3+3=6$是表达正确的数学语句,而3+=3\$6 则不是。$H_2O$是表达正确的化学式,而$_2Z_Z$则不是。

语法规则有两种形式,从属于标记和结构。标记是语言的基本元素,例如单词、数字以及化学元素。3+=3\$6 的一个问题是\$ 不是合法的数学标记。类似地,$_2Z_Z$也不是合法,因为没有$Z_Z$元素的缩写。

第二类语法错误属于句式结构上的。即标记的排列规则语句3+=3\$6不是合法的,尽管+和=是合法的,但是不能紧跟在一起。类似地在化学式中,下标符号位于化学元素名的后面,而不能在前面。
\begin{exercise}
写出一个结构良好的英语句子但中间加入了一个无效的标记。写一个全部由有效标记结成的但结构不合法的句子。
\end{exercise}
当你阅读一个用英语写的句子或者一个用形式语言写的句子时,你必须理解语句的结构是什么(尽管对于自然语言,这个过程是在潜意识中进行的)。该过程称为语法分析。

例如:当阅读句子 “The penny droped," 你能够理解 “the penny" 是主语而 “drop\-ped" 是谓语。一旦从语法上对句子进行了分析,你将能够理解句子是什么意思,或者句子的语义。假如知道一“美分”是什么意以及“掉下”是什么意思,你将明白句子的通常含义。

尽管形式语言和自然语言在标记、结构、语法和语义上有许多共同之处。其不同之和在于:
\begin{description}
\item[二义性不同:]自然语言是完全模糊的,取决于人对上下文和其它信息的处理。形式语文被设计成为几乎或才完全无二义的,不管上下文如何,任何句子都确切的只表示一个意义。
\item[冗余度不同:]为了弥补二义和减少错误理解,自然语言使用很多的重复,其结果必然导致冗长。形式语言很少重复并且更加便利。
\item[字面意义不同:]自然语言充满着习惯用语和隐喻。如果说:“目的达到了”,这里既没有钱也没有东西掉了。形式语言精确的表达其意思。
\end{description}
惯于说自然语言的人经常有难于适应形式语言。从某方面来说,形式放言与自然语言之间的区别就如同诗歌与散文之间的区别一样,甚至更多:
\begin{description}
\item[诗歌:]为其曲调和意思而使用单词,整首诗一起构建了效果或者情感的共鸣。二义性不仅是普遍的而且还是故意为之。
\item[散文:]单词的字面意义更加重要,并且结构也贡献更多的意义。散文比诗歌更经得起分析,但也经常是二义的。
\item[程序:]计算机程序是无二义的和字面的,能够进行标识和结构分析。
\end{description}
在此有一些阅读程序(和其它形式语言)的建议。首先,记住形式语言比自然语言更加紧密,因此阅读它要更长的时间。并且其结构也很非常重要,因此从上到下、从左到右的阅读方式并不是一个好主意。反而,学习在头脑中对程序进行语法上的分析,识别其标识和解释其结构。最后,着重于细节问题。在自然语言中我们可能不必计较的小的拼写错误或者标点符号,在形式语言中将会产生很大的不同。

\section{第一个程序}
习惯上,你所写的第一个程序被称作“Hello,World!”。因为在 Python 中它只显示“Hello,World!”。像这样:
\begin{lstlisting}
print 'Hello, World!'
\end{lstlisting}

这是一个打印语句的例子,它并不纸上打印任何东西,它在显示屏上显示一个值。既然这样,结果是:
\begin{lstlisting}
Hello, World!
\end{lstlisting}
单引号在程序中标识显示文本的开始和结束,它们不在结果中显示。
 
一些人通过“Hello, World!”的简洁程序来判断程序语言的品质。基于这个标准,Py\-thon 可能是其中之一。
\section{调试}
在计算机前面阅读本书是一个不错的主意,因此你可以试验你看到的例子。你能够在交互模式下运行大多数的例子,若将代码放到一脚本中去,就能够很容易的进行多次实验。

无论什么时候试验一个新特征,你应当试图产生错误。例如,在“Hello, World!”程序中,如果你遗漏一个引号将发生什么?遗漏两呢?print 拼写错误呢?

这类尝试能够帮助你记住好些你所读到的东西,也能够帮助你做调度,因为你能够知道错误信息的含义是什么。现在有意的制造错误好于后面意外的错误。

编程和特别的调度,往往带有强烈的感情。如果与一个困难的错误作斗争,你也许有生气的、失望的或者窘的感觉。

有证据表明人们自然地回应计算机,好像它们是人一样的。当工作顺利的时候,我们认为它是队伍,当他们是顽固的或者是,我们也用同样的方式予以回应。

为这些反应做好准备能有助于我们处理它们。一种方法是将计算机看作一个有员工,喜欢速度与精确,特别脆弱,缺乏共鸣,无全局观念。

你的工作是一个管理者:寻找各种方式发挥其长处,克服其不足。并寻找用你激情去迎接问题的方式,使你的反应没有干扰你的工作效率的能力。

学习调试可能令人沮丧,但它是用于很多编程活动中最有价值的技能。像本章一样,每一章结束的地方都是我关于调试思考的部分。希望它们对你有用。

\endinput