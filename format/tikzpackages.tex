\usepackage{CJKutf8,CJKnumb}
\usepackage{titlesec}
\usepackage{titletoc}
\usepackage{fancyhdr}
\usepackage{makeidx}
\makeindex
\usepackage[paperwidth=185mm,paperheight=260mm,text={148mm,210mm},
			left=21mm,vmarginratio=1:1]{geometry}
\usepackage{xcolor}
\usepackage{listings}
\newif\ifpdf
\ifx\pdfoutput\undefined
   \pdffalse
\else
   \pdfoutput=1
   \pdftrue
\fi
\ifpdf
   \usepackage[pdftex]{graphicx}
   \usepackage[pdftex,unicode=true]{hyperref}
\else
   \usepackage{graphicx}
   \usepackage[unicode={true}]{hyperref}
\fi
\usepackage{tikz}

% We need lots of libraries...
\usetikzlibrary{
  arrows,
  calc,
  fit,
  patterns,
  plotmarks,
  shapes.geometric,
  shapes.misc,
  shapes.symbols,
  shapes.arrows,
  shapes.callouts,
  shapes.multipart,
  shapes.gates.logic.US,
  shapes.gates.logic.IEC,
  circuits.logic.US,
  circuits.logic.IEC,
  circuits.logic.CDH,
  circuits.ee.IEC,
  datavisualization,
  datavisualization.formats.functions,
  er,
  automata,
  backgrounds,
  chains,
  topaths,
  trees,
  petri,
  mindmap,
  matrix,
  calendar,
  folding,
  fadings,
  shadings,
  spy,
  through,
  turtle,
  positioning,
  scopes,
  decorations.fractals,
  decorations.shapes,
  decorations.text,
  decorations.pathmorphing,
  decorations.pathreplacing,
  decorations.footprints,
  decorations.markings,
  shadows,
  lindenmayersystems,
  intersections,
  fixedpointarithmetic,
  fpu,
  svg.path,
  external,
}

\iffalse
%\iftrue
	\tikzexternalize[
		mode=list only,export=true,% simply skips EVERY picture -> good for debugging the text.
	]
		{pgfmanual}

	\tikzifexternalizing{%
		\pgfkeys{/pdflinks/codeexample links=false}%
	}{}%
\fi
\usepackage{indentfirst}
\setlength{\parindent}{2em}
\tikzset{
  every plot/.style={prefix=plots/pgf-},
  shape example/.style={
    color=black!30,
    draw,
    fill=yellow!30,
    line width=.5cm,
    inner xsep=2.5cm,
    inner ysep=0.5cm}
}
\endinput