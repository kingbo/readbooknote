\chapter{初识Go语言}
Go语言的目标之一是一致且无歧义的语法。这个特点利于使用工具来检测GO语言程序,也使得它易于学习。无用的编译错误增加了学习程序语言的难度,这点对于他用过C++模板代码并产生的打字排版错误的人是十分清楚的。

例如在C语言中函数变量和全局变量声明的语法几乎是一样的。如果你出错将意味着编译器不能够顺利地的告诉你哪一个是你所想要的。编译器在某行给出一条件像“expected ;”一样有帮助的错误信息,而那一行却不总是需要分号。

Go语言的语法器被设计成认编译器能够给出更准确的错误信息。同时它也被设计成避免依赖某种状态,因此更加容易推断错误。例如,如果你建立了一个变量并将其设置成42,编译器能够在没有明确的状态下大概猜出该变量是整型变量。如果你在函数调用中初始化该变量,编译器能够明确该变量的类型,无论函数在哪里返回。这点和C++2011的auto类型一样。

Go语言采用采用了JavaScript的分号插入思想,且比它更先进。任何行都记法分析器解释成为行尾有隐式分号插入的的完整语句。这意味着Go语言能够直接忽略用分号作为语句终结符。因此,也增加了一些限制,如强制括号样式为左括号在控制语句开始行末,而不是我们自己。碰巧你是这样的人,非常不幸,这意味着你不能使用最优的系统识别方式,为了理解代码块,视觉皮层估计需要至少需要花费无数年才能够完成优化。

本章只是Go语言的语法概览,并不包含全部的内容。某些方面在后续章节中覆盖。实际上,Go语言所有与并发相关的内容包含在第9章,Goroutines中。

\section{Go语言源文件结构}
\begin{lstlisting}
package main
import "fmt"
func main(){
   fmt.Printf("Hello World!\n")
}
\end{lstlisting}

Go语言源文件由三个部分组成。第一个部分是package语句。Go代码被放置在包中,包在Go语言中起类似C语言中的库和头文件的作用。本例中的包名是main,它是一个特殊的包。每一个Go程序必须包含一个main包,并包含一个main()函数,它是程序的入口。

第二部分是明确该源文件所要用到的包,以及如何导入它们。在上例中,我们导入了fmt包。一旦导入fmt包,冠以包的字首就能够使用该包所定义的导出类型、变量、常数和函数。在上例中,我们调用了与C语言的printf函数类似的Printf()函数,并用它在终端上输出“Hello World!”字样。

尽管Go语言使用静态编译,但是,明白import语句更接近于Java或Python的导入指令而不是C语言的include指令是很重要的,因此它们并不将源代码包含于当前的编译单元。不像Java和Python的包,Go语言的包将在代码连接的时候导入而不等到运行的时候。这样确保了Go语言的应用程序不会因为布署系统缺少相应的包而失败,但代价是增加了可执行程序文件的大小。包在Go语言中比类似Java语言的包更加重要。因为Go语言仅提供包级层次的进入控制,而Java提供到类级的层。

当使用GC编译器编译一个包时,我们将从包获得目标代码文件。它包含一个用以描述包输出类型和函数的元数据。同时也包含了该包所导入的包列表。

对main包调用6l连接器总是获得a.t文件。该文件包括了在main包中导入的每一个包的起始位置,也会依次包含更多包的起始位置。到时连接器会将所有的组合起来。

建立复杂C语言程序最令人不愉快的限制是:你要包含一个头文件,并给出它提供的库和连接标志。在Go语言中,如果是包编译,它将会连接,你不必提供连接标志给连接器,并通过import指令告诉连接器你的引用。

Go语言中余下的类型、变量和函数声明,将在本章的其余部分进行阐述。

你也许发现您想导入的两个包有相同的名字。这在Go语言中将引发问题。例子badStyleImport.go 是一个与本节开始的例子功能相同,但将fmt包名重命名为format。在导入包的时候进行重命名通常是个坏主意,因为它使得代码难以被人阅读。你仅仅在确实需要消除两个同名包的二义性时才使用。

\begin{lstlisting}
package main
import format "fmt"

func main(){
	format.Printf("Hello World!\n")
}
\end{lstlisting}

\section{变量定义}
\begin{lstlisting}
var i int

var eplicitly,typed, pointers *complex128
int_pointer :=&i 
another_int_point :=new(int)
generic_channel :=make(chan interface{})
\end{lstlisting}
\endinput