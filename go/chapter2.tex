\chapter{初识Go语言}
Go语言的目标之一是一致且无歧义的语法。这个特点利于使用工具来检测GO语言程序,也使得它易于学习。无用的编译错误增加了学习程序语言的难度,这点对于他用过C++模板代码并产生的打字排版错误的人是十分清楚的。

例如在C语言中函数变量和全局变量声明的语法几乎是一样的。如果你出错将意味着编译器不能够顺利地的告诉你哪一个是你所想要的。编译器在某行给出一条件像“expected ;”一样有帮助的错误信息,而那一行却不总是需要分号。

Go语言的语法器被设计成认编译器能够给出更准确的错误信息。同时它也被设计成避免依赖某种状态,因此更加容易推断错误。例如,如果你建立了一个变量并将其设置成42,编译器能够在没有明确的状态下大概猜出该变量是整型变量。如果你在函数调用中初始化该变量,编译器能够明确该变量的类型,无论函数在哪里返回。这点和C++2011的auto类型一样。

Go语言采用采用了JavaScript的分号插入思想,且比它更先进。
\endinput